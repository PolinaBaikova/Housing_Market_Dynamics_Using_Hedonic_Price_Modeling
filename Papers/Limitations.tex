\section*{Limitations and Considerations}

Although the hedonic pricing model developed in this study offers meaningful insights into the valuation of housing attributes, several methodological limitations should be acknowledged. These limitations help frame the interpretation of results and point to directions for future research.

As is typical in housing market analyses, it is challenging to account for all factors that influence home prices. Important attributes such as recent interior renovations, neighborhood prestige, local crime rates, and buyer sentiment were not included due to data limitations or measurement difficulties. The omission of such variables introduces the possibility of omitted variable bias if they are correlated with observed features.

Some predictors in the model may also be endogenous, shaped in part by expectations about resale value or broader market dynamics. For instance, the decision to add a pool or garage may reflect anticipated returns, and school quality can respond dynamically to neighborhood demographics. This raises potential concerns about reverse causality affecting coefficient estimation.

The dataset spans home sales from 2023 to 2025, a period characterized by post-pandemic adjustments in buyer preferences, supply constraints, and macroeconomic uncertainty. As such, the findings should be interpreted as reflecting short- to medium-term market conditions rather than long-run structural relationships.

In addition, hedonic models assume that housing prices reflect equilibrium outcomes, with informed buyers and sellers fully capitalizing the value of property features into transaction prices. However, real-world market frictions, behavioral biases, and temporary supply-demand imbalances may cause deviations from this ideal.

To build on this work, future research could incorporate a broader set of property- and neighborhood-level variables, apply techniques such as instrumental variables to address endogeneity, extend the time frame to assess temporal stability, and explore alternative modeling strategies such as spatial econometrics or machine learning to improve predictive performance and causal interpretation.

Despite these limitations, the model developed here serves as a useful baseline for understanding how housing attributes are priced across Orlando’s diverse submarkets.
