\section*{Institutional Details}
The residential real estate market in Florida is shaped by a combination of institutional, regulatory, and economic factors that influence both the supply and the demand. The state’s lack of personal income tax and relatively low property taxes have contributed to steady in-migration from other states. Within Florida, the Orlando Metropolitan Area stands out as a key destination due to its economic growth, entertainment infrastructure, and expanding job market.

Orlando’s housing market has undergone substantial transformation over the past decade ($2014$ -- $24$). During this period, the median home price nearly doubled --- from around $\$160,000$ to approximately $\$380,000$ \citep{orra:2024}. Even though part of this growth reflects inflation, a significant portion represents real appreciation driven by sustained demand and population growth. Adjusted for inflation, the 2014 median price would be about $\$257,909$ in $2024$ dollars, indicating notable real gains in property values \citep{bls:2024}.
The Orlando housing market is characterized by a diverse range of property types, from suburban single-family homes to urban condominiums and newly developed subdivisions. 

Local governments, including the county and the city planning boards, regulate land use through zoning ordinances that affect housing density, lot sizes, and permitted property uses. These zoning regulations, along with school district boundaries and proximity to transportation infrastructure, contribute to spatial variations in housing prices across the region.
Home purchases in Florida are typically facilitated through Multiple Listing Services, and prices are determined in a decentralized, market-based system where buyer preferences are revealed through negotiated sales. Property values are also affected by policies such as the Save Our Homes amendment, which limits annual increases in assessed value for owner-occupied homes, and by seasonal factors that drive demand fluctuations throughout the year.

A comprehensive understanding of housing market trends allows homebuyers to make informed purchasing decisions, enables developers to anticipate demand and allocate resources efficiently, assists investors in identifying areas with strong growth potential, and supports policymakers in formulating effective strategies for land use, infrastructure planning, and housing affordability. A rigorous analysis of the factors influencing housing prices is essential for guiding evidence-based decisions in a rapidly evolving urban environment.
