\section*{Introduction and Motivation}

In Florida, housing demand has increased significantly. The Orlando Metropolitan Area has emerged as one of the fastest-growing regions in the United States, leading Florida’s growth statistics in recent years. Forecasts suggest the population may increase by nearly one million by 2045, presenting challenges for housing supply, infrastructure development, and urban planning \citep{bebr:2023}. 

To address these challenges and to inform sound policy and investment decisions, a deeper understanding of the housing market dynamics in Orlando is essential. When exploring housing prices, it is important to account for the full range of characteristics that influence housing prices --- not only physical attributes of a property itself, but neighborhood and locational factors, too.

The Hedonic pricing offers a rigorous, data-driven approach to decompose prices into the implicit values of their underlying characteristics \citep{rosen:1974}. By applying it to real estate and analyzing actual transaction data, this project aims to estimate homebuyers’ valuations of various housing features. The goal is to develop a flexible empirical model that captures how housing attributes influence prices in the Orlando region, offering insight into the structure of demand in this growing urban market and providing a framework that can be extended to other geographic areas.
