\section*{Conclusions}

This study employed hedonic pricing methods to analyze residential property valuations in Orlando, focusing on how structural attributes, neighborhood characteristics, and locational factors influence home prices across Central and East Orlando. Using cross-validated least squares regression and GAM, the analysis the analysis explored how various housing features correlate with market value, while accounting for regional heterogeneity and nonlinear effects.

The findings illustrate that attributes such as interior space, amenity features, school quality, flood risk status, proximity to highways, and property age significantly affect sale prices, with notable regional differences reflecting distinct urban and suburban buyer preferences. In Central Orlando, denser urban development underscores buyer demand for efficient space utilization, accessibility, and proximity to services, while East Orlando’s suburban character elevates the value of privacy, exclusivity, and lower-density environments.

By incorporating GAMs, the analysis captured important nonlinear dynamics, such as diminishing returns to space and nonlinear effects of school quality and proximity variables, which linear models often fail to detect. Derivative-based marginal effect estimation further revealed how the influence of continuous features shifts across the distribution, providing a more nuanced view of buyer preferences.


Overall, the study provides an empirically robust foundation for understanding Orlando’s housing market dynamics, offering valuable insights and practical recommendations for homebuyers, sellers, developers, and policymakers. By integrating nonlinear modeling and region-specific analysis, this work highlights the importance of tailoring housing strategies to local preferences and encourages future research to build on this foundation using spatial, temporal, and causal modeling enhancements.


