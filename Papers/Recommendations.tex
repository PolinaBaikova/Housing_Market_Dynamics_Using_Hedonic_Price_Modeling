\section*{Practical Recommendations}

The results of this analysis have practical implications for various stakeholders in the real estate market:

\begin{quote} 

\noindent \textbf{Home buyers:}

\noindent \textit{Make Informed Investment Decisions.} Buyers can use the estimated price effects to spot investment opportunities. A home with features such as a pool, garage, or high school ratings listed below expected value may signal underpricing. In Central Orlando, prioritize space-efficient interiors and parking access; in East Orlando, prioritize homes that offer privacy, recreational amenities, and a quieter residential setting.

\noindent \textit{Avoid Overpaying for Low-Value Attributes.}
\noindent Certain features carry negative or negligible value. Multi-story layouts in East Orlando are associated with lower prices, while HOA restrictions reduce value in both regions. Buyers should factor these effects into their offer and weigh the long-term costs against listed price.

\noindent \textit{Consider Long-Term Resale Potential.}
\noindent Even if certain features, such as proximity to a park, are not a personal priority, their influence on future resale value should not be overlooked.  Homes that score well on high-value attributes tend to maintain or increase value more reliably. 


\noindent \textbf{Sellers:}

\noindent \textit{Highlight High-Value Features.}
In marketing and showings, emphasize attributes that command strong premiums. In Central Orlando, focus on interior space, number of bedrooms and bathrooms, and garage access. In East Orlando, spotlight features like private pools, gated entry, and single-story layout.

\noindent \textit{Make Targeted, Cost-Effective Improvements.}
Renovate selectively. In Central Orlando, consider interior upgrades and additional space. In East Orlando, enhancing outdoor areas or ensuring ease of access for older buyers may yield stronger returns. 

\noindent \textit{Offset Negative Factors.}
If a property lacks key value drivers, it is important to emphasize compensating strengths or propose thoughtful alternatives. For instance, in the absence of a garage, highlighting off-street parking availability can address buyer concerns. Similarly, if proximity to parks or scenic views is limited, showcasing private outdoor areas, quality landscaping, or access to nearby recreational amenities can help convey comparable lifestyle advantages.

\noindent \textbf{Real Estate Developers and Investors:}

\noindent \textit{Incorporate Desirable Site Characteristics.}
Enhance project value by leveraging environmental and locational attributes. In urban developments, proximity to parks and green spaces increases buyer appeal. In flood-prone areas, implement mitigation strategies such as elevated construction and modern drainage to reduce risk. Ensure access to highways or transit, but buffer residential units from traffic-related noise and congestion.

\noindent \textit{Use Data-Driven Strategies.}
Apply hedonic pricing insights to optimize development plans and capital allocation. In compact markets, maximize usable square footage per lot; in suburban settings, offer larger parcels with community features like security gates, walking trails, or shared recreational areas. Grounding design choices in local pricing patterns ensures higher return on investment and market alignment.



\noindent \textbf{City Planners and Policy Makers:}

\noindent \textit{Invest in Public Amenities that Raise Property Values.}
Expanding access to green space in urban areas and improving school outcomes raises surrounding property values. Since property taxes are often based on assessed value, investments in amenities such as parks can directly increase municipal revenue by lifting local home prices.

\noindent \textit{Support Resilient Infrastructure and Risk-Responsive Zoning.}
\noindent Promote land use and infrastructure policies that reflect market preferences and mitigate environmental risks. Invest in drainage systems and flood protection to safeguard property values in vulnerable areas. Encourage zoning flexibility and incentivize development in urban cores to make efficient use of limited land.

\noindent \textit{Improve Urban Transportation and Access.}
\noindent Expand pedestrian, bicycle, and transit infrastructure in city centers to reduce reliance on cars and improve accessibility. Providing convenient access to schools, parks, and commercial services helps sustain long-term property demand.

\end{quote}
