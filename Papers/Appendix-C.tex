\subsection*{Appendix C}

To identify closely-related predictors, I computed pairwise Pearson correlations. In Table~\ref{tab:high_corr}, I present variable pairs with absolute correlation coefficients greater than 0.50.



Distances to highways, distances to schools, and school ratings exhibited high pairwise correlations. To address this, I introduced a set of composite and derived variables. 

First, I created \texttt{min\_distance\_highway}, defined as the distance from each home to its nearest major highway exit, to replace individual highway distance variables. Second, I aggregated school-related information into two summary variables: \texttt{avg\_schools\_rating}, the mean rating across primary, middle, and high schools; and \texttt{avg\_distance\_to\_schools}, the average distance to each of the three closest schools by level. Finally, I transformed \texttt{living\_area} into \texttt{sqft\_per\_bedroom}, a functional measure of interior space per room that reduces redundancy between size-related variables while preserving interpretability.

\begin{table}
\caption{Pairs of Variables with High Correlation}
\label{tab:high_corr}
\begin{tabular}{llr}
\toprule
Variable 1 & Variable 2 & Correlation \\
\midrule
living\_area & bathrooms & 0.76 \\
parking\_spaces & has\_garage & 0.75 \\
living\_area & bedrooms & 0.75 \\
middle\_school\_rating & dist\_to\_i4 & 0.67 \\
view & water\_view & 0.63 \\
dist\_to\_sr417 & dist\_to\_sr528 & 0.63 \\
bathrooms & levels & 0.63 \\
age & dist\_to\_i4 & -0.61 \\
middle\_school\_rating & high\_school\_rating & 0.60 \\
living\_area & parking\_spaces & 0.60 \\
middle\_school\_rating & dist\_to\_sr408 & 0.60 \\
park & playground & 0.58 \\
age & hoa & -0.58 \\
bedrooms & bathrooms & 0.56 \\
dist\_to\_i4 & dist\_to\_sr408 & 0.55 \\
home\_type & bedrooms & 0.52 \\
primary\_school\_rating & middle\_school\_rating & 0.52 \\
age & middle\_school\_rating & -0.51 \\
primary\_school\_rating & dist\_to\_sr408 & 0.51 \\
home\_type & parking\_spaces & 0.51 \\
dist\_to\_i4 & dist\_to\_sr417 & -0.50 \\
\bottomrule
\end{tabular}
\end{table}

